% Resume autogenerated using resume-latex.xsl

\documentclass[letterpaper]{resume}
\usepackage{paralist}
\definecolor{ruleendcolor}{rgb}{0.6, 0.6, 0.6}


    
\author{Daniel Burrows}
\streetaddress{402 Nagle St. Apt 307}
\citystatezip{College Station, TX, 77840}
    
\email{dburrows@debian.org}

    
\phone{(814)360-2175}

    
\begin{document}
\maketitle

\section{Education}

\affiliation[Bachelor of Arts in Computer Science, Magna Cum Laude]
            {Brown University}
            {September 1998-May 2002}

\begin{compactitem}

	\item GPA: 3.55 overall, 3.81 in Computer Science \par
      
\end{compactitem}

\affiliation[Master of Science in Computer Science and Engineering]
            {The Pennsylvania State University}
            {September 2002-May 2005}

\begin{compactitem}

	\item Wrote thesis: ``Static Encapsulation Analysis in Featherweight Java''. \par

	\item GPA: 3.80 \par
      
\end{compactitem}

\section{Experience}

\affiliation[Debian Developer]
            {Debian Project (\url{http://www.debian.org})}
            {October 2000-present}

\begin{compactitem}

	\item Worked with the Debian Project, a worldwide volunteer organization dedicated to producing a high-quality, free, Linux-based operating system. \par

	\item Created the aptitude package management frontend (approximately 47,000 lines of C++ code and 10,000 lines of DocBook XML documentation)
      
\end{compactitem}

\affiliation[Graduate Teaching Assistant]
            {The Pennsylvania State University \par Department of Computer Science and Engineering \par (\url{http://www.cse.psu.edu})}
            {September 2002-December 2003}

\begin{compactitem}

	\item Graded and assisted students in CSE 468, Theory of Formal Languages. \par

	\item Performed grading duties and prepared course materials, including homeworks, supplementary handouts, and a BSP editor, for CSE 418, Computer Graphics. \par
      
\end{compactitem}

\affiliation[Summer Internship through NASA's Education Associates program]
            {NASA AMES Research Center \par (\url{http://www.arc.nasa.gov})}
            {June 2003-August 2003}

\begin{compactitem}

	\item Worked on the Livingstone project, a model-based engine for automated diagnosis of and recovery from hardware failures on spacecraft. \par

	

	\item Studied and implemented techniques for artificial intelligence, particularly automated logical inference ($A^\star$ search, etc). \par
      
\end{compactitem}

\affiliation[Assistant MOC Network Administrator]
            {Penn State/NASA Swift Project \par Department of Astronomy and Astrophysics \par (\url{http://www.swift.psu.edu})}
            {January 2004-June 2005}

\begin{compactitem}

       \item Installed and configured Debian GNU/Linux science support workstations and a central NFS/LDAP server on the Swift Mission Operations Center network. \par

       \item Wrote Python scripts to assist in maintaining a network of identical workstations. \par
     
\end{compactitem}

\newpage\section{Skills}


\begin{compactitem}

      \item Programming in languages including C++, Python, and Lisp. \par C++: \par \begin{compactitem} \item 7 years experience. \par \item Wrote a 45,000 line frontend application to manage Debian software packages. \par \item Wrote a solver for systems of logical constraints at NASA Ames. \par \item Spent a semester testing, reading and grading C++-based computer graphics projects. \par \end{compactitem} Python: \par \begin{compactitem} \item 6 years experience. \par \item Wrote a 3,000 line graphical music organizer using the GTK+ graphical user interface library. \par \item Wrote a relay-only mail transport agent. \par \item Implemented a system for synchronizing configuration changes across multiple workstations. \par \item Wrote Python bindings for the ``auto-mark'' extension to the ``apt'' C++ library. \par \end{compactitem}

      

      \item Document creation in \LaTeX\ and in Docbook XML. \par

      \item 7 years \textsc{Unix/Linux} experience, including software development and system administration (primarily Debian GNU/Linux). \par
    
\end{compactitem}


\end{document}
